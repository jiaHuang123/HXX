%%
%DIF LATEXDIFF DIFFERENCE FILE
%DIF DEL poster.tex     Fri Sep 18 13:58:30 2020
%DIF ADD posterHJ.tex   Tue Oct 13 22:43:17 2020
%% This is file `tikzposter-template.tex',
%% generated with the docstrip utility.
%%
%% The original source files were:
%%
%% tikzposter.dtx  (with options: `tikzposter-template.tex')
%%
%% This is a generated file.
%%
%% Copyright (C) 2014 by Pascal Richter, Elena Botoeva, Richard Barnard, and Dirk Surmann
%%
%% This file may be distributed and/or modified under the
%% conditions of the LaTeX Project Public License, either
%% version 2.0 of this license or (at your option) any later
%% version. The latest version of this license is in:
%%
%% http://www.latex-project.org/lppl.txt
%%
%% and version 2.0 or later is part of all distributions of
%% LaTeX version 2013/12/01 or later.
%%


\documentclass{tikzposter} %Options for format can be included here

\usepackage{todonotes}

\usepackage[tikz]{bclogo}
\usepackage{lipsum}
\usepackage{amsmath}

\usepackage{booktabs}
\usepackage{longtable}
\usepackage[absolute]{textpos}
\usepackage[it]{subfigure}
\usepackage{graphicx}
\usepackage{cmbright}
%\usepackage[default]{cantarell}
%\usepackage{avant}
%\usepackage[math]{iwona}
\usepackage[math]{kurier}
\usepackage[T1]{fontenc}
%DIF 44c44
%DIF < 
%DIF -------
\usepackage{graphicx} %DIF > 
%DIF -------

%% add your packages here
\usepackage{hyperref}
% for random text
\usepackage{lipsum}
\usepackage[english]{babel}
\usepackage[pangram]{blindtext}

\colorlet{backgroundcolor}{blue!10}file:///D:/SmartGitSTR/templatex/logos

 % Title, Author, Institute
\title{\DIFdelbegin \DIFdel{Group Outlying Aspects Mining}\DIFdelend \DIFaddbegin \DIFadd{San Francisco Crime Classification}\DIFaddend }
\author{\DIFdelbegin \DIFdel{Shaoni Wang$^1$, Gang Li$^2$}\DIFdelend \DIFaddbegin \DIFadd{Jia Huang}\DIFaddend }
\institute{$^1$ Xi'an Shiyou University, China \\
\DIFdelbegin \DIFdel{$^2$ Deakin University, Australia
}\DIFdelend }
%\titlegraphic{logos/tulip-logo.eps}

%Choose Layout
\usetheme{Wave}

%\definebackgroundstyle{samplebackgroundstyle}{
%\draw[inner sep=0pt, line width=0pt, color=red, fill=backgroundcolor!30!black]
%(bottomleft) rectangle (topright);
%}
%
%\colorlet{backgroundcolor}{blue!10}
%DIF PREAMBLE EXTENSION ADDED BY LATEXDIFF
%DIF UNDERLINE PREAMBLE %DIF PREAMBLE
\RequirePackage[normalem]{ulem} %DIF PREAMBLE
\RequirePackage{color}\definecolor{RED}{rgb}{1,0,0}\definecolor{BLUE}{rgb}{0,0,1} %DIF PREAMBLE
\providecommand{\DIFaddtex}[1]{{\protect\color{blue}\uwave{#1}}} %DIF PREAMBLE
\providecommand{\DIFdeltex}[1]{{\protect\color{red}\sout{#1}}}                      %DIF PREAMBLE
%DIF SAFE PREAMBLE %DIF PREAMBLE
\providecommand{\DIFaddbegin}{} %DIF PREAMBLE
\providecommand{\DIFaddend}{} %DIF PREAMBLE
\providecommand{\DIFdelbegin}{} %DIF PREAMBLE
\providecommand{\DIFdelend}{} %DIF PREAMBLE
\providecommand{\DIFmodbegin}{} %DIF PREAMBLE
\providecommand{\DIFmodend}{} %DIF PREAMBLE
%DIF FLOATSAFE PREAMBLE %DIF PREAMBLE
\providecommand{\DIFaddFL}[1]{\DIFadd{#1}} %DIF PREAMBLE
\providecommand{\DIFdelFL}[1]{\DIFdel{#1}} %DIF PREAMBLE
\providecommand{\DIFaddbeginFL}{} %DIF PREAMBLE
\providecommand{\DIFaddendFL}{} %DIF PREAMBLE
\providecommand{\DIFdelbeginFL}{} %DIF PREAMBLE
\providecommand{\DIFdelendFL}{} %DIF PREAMBLE
%DIF HYPERREF PREAMBLE %DIF PREAMBLE
\providecommand{\DIFadd}[1]{\texorpdfstring{\DIFaddtex{#1}}{#1}} %DIF PREAMBLE
\providecommand{\DIFdel}[1]{\texorpdfstring{\DIFdeltex{#1}}{}} %DIF PREAMBLE
\newcommand{\DIFscaledelfig}{0.5}
%DIF HIGHLIGHTGRAPHICS PREAMBLE %DIF PREAMBLE
\RequirePackage{settobox} %DIF PREAMBLE
\RequirePackage{letltxmacro} %DIF PREAMBLE
\newsavebox{\DIFdelgraphicsbox} %DIF PREAMBLE
\newlength{\DIFdelgraphicswidth} %DIF PREAMBLE
\newlength{\DIFdelgraphicsheight} %DIF PREAMBLE
% store original definition of \includegraphics %DIF PREAMBLE
\LetLtxMacro{\DIFOincludegraphics}{\includegraphics} %DIF PREAMBLE
\newcommand{\DIFaddincludegraphics}[2][]{{\color{blue}\fbox{\DIFOincludegraphics[#1]{#2}}}} %DIF PREAMBLE
\newcommand{\DIFdelincludegraphics}[2][]{% %DIF PREAMBLE
\sbox{\DIFdelgraphicsbox}{\DIFOincludegraphics[#1]{#2}}% %DIF PREAMBLE
\settoboxwidth{\DIFdelgraphicswidth}{\DIFdelgraphicsbox} %DIF PREAMBLE
\settoboxtotalheight{\DIFdelgraphicsheight}{\DIFdelgraphicsbox} %DIF PREAMBLE
\scalebox{\DIFscaledelfig}{% %DIF PREAMBLE
\parbox[b]{\DIFdelgraphicswidth}{\usebox{\DIFdelgraphicsbox}\\[-\baselineskip] \rule{\DIFdelgraphicswidth}{0em}}\llap{\resizebox{\DIFdelgraphicswidth}{\DIFdelgraphicsheight}{% %DIF PREAMBLE
\setlength{\unitlength}{\DIFdelgraphicswidth}% %DIF PREAMBLE
\begin{picture}(1,1)% %DIF PREAMBLE
\thicklines\linethickness{2pt} %DIF PREAMBLE
{\color[rgb]{1,0,0}\put(0,0){\framebox(1,1){}}}% %DIF PREAMBLE
{\color[rgb]{1,0,0}\put(0,0){\line( 1,1){1}}}% %DIF PREAMBLE
{\color[rgb]{1,0,0}\put(0,1){\line(1,-1){1}}}% %DIF PREAMBLE
\end{picture}% %DIF PREAMBLE
}\hspace*{3pt}}} %DIF PREAMBLE
} %DIF PREAMBLE
\LetLtxMacro{\DIFOaddbegin}{\DIFaddbegin} %DIF PREAMBLE
\LetLtxMacro{\DIFOaddend}{\DIFaddend} %DIF PREAMBLE
\LetLtxMacro{\DIFOdelbegin}{\DIFdelbegin} %DIF PREAMBLE
\LetLtxMacro{\DIFOdelend}{\DIFdelend} %DIF PREAMBLE
\DeclareRobustCommand{\DIFaddbegin}{\DIFOaddbegin \let\includegraphics\DIFaddincludegraphics} %DIF PREAMBLE
\DeclareRobustCommand{\DIFaddend}{\DIFOaddend \let\includegraphics\DIFOincludegraphics} %DIF PREAMBLE
\DeclareRobustCommand{\DIFdelbegin}{\DIFOdelbegin \let\includegraphics\DIFdelincludegraphics} %DIF PREAMBLE
\DeclareRobustCommand{\DIFdelend}{\DIFOaddend \let\includegraphics\DIFOincludegraphics} %DIF PREAMBLE
\LetLtxMacro{\DIFOaddbeginFL}{\DIFaddbeginFL} %DIF PREAMBLE
\LetLtxMacro{\DIFOaddendFL}{\DIFaddendFL} %DIF PREAMBLE
\LetLtxMacro{\DIFOdelbeginFL}{\DIFdelbeginFL} %DIF PREAMBLE
\LetLtxMacro{\DIFOdelendFL}{\DIFdelendFL} %DIF PREAMBLE
\DeclareRobustCommand{\DIFaddbeginFL}{\DIFOaddbeginFL \let\includegraphics\DIFaddincludegraphics} %DIF PREAMBLE
\DeclareRobustCommand{\DIFaddendFL}{\DIFOaddendFL \let\includegraphics\DIFOincludegraphics} %DIF PREAMBLE
\DeclareRobustCommand{\DIFdelbeginFL}{\DIFOdelbeginFL \let\includegraphics\DIFdelincludegraphics} %DIF PREAMBLE
\DeclareRobustCommand{\DIFdelendFL}{\DIFOaddendFL \let\includegraphics\DIFOincludegraphics} %DIF PREAMBLE
%DIF LISTINGS PREAMBLE %DIF PREAMBLE
\RequirePackage{listings} %DIF PREAMBLE
\RequirePackage{color} %DIF PREAMBLE
\lstdefinelanguage{DIFcode}{ %DIF PREAMBLE
%DIF DIFCODE_UNDERLINE %DIF PREAMBLE
  moredelim=[il][\color{red}\sout]{\%DIF\ <\ }, %DIF PREAMBLE
  moredelim=[il][\color{blue}\uwave]{\%DIF\ >\ } %DIF PREAMBLE
} %DIF PREAMBLE
\lstdefinestyle{DIFverbatimstyle}{ %DIF PREAMBLE
	language=DIFcode, %DIF PREAMBLE
	basicstyle=\ttfamily, %DIF PREAMBLE
	columns=fullflexible, %DIF PREAMBLE
	keepspaces=true %DIF PREAMBLE
} %DIF PREAMBLE
\lstnewenvironment{DIFverbatim}{\lstset{style=DIFverbatimstyle}}{} %DIF PREAMBLE
\lstnewenvironment{DIFverbatim*}{\lstset{style=DIFverbatimstyle,showspaces=true}}{} %DIF PREAMBLE
%DIF END PREAMBLE EXTENSION ADDED BY LATEXDIFF

\begin{document}


\colorlet{blocktitlebgcolor}{blue!23}

 % Title block with title, author, logo, etc.
\maketitle

\begin{columns}
 % FIRST column
\column{0.5}% Width set relative to text width

%%%%%%%%%% -------------------------------------------------------------------- %%%%%%%%%%
 %\block{Main Objectives}{
%  	      	\begin{enumerate}
%  	      	\item Formalise research problem by extending \emph{outlying aspects mining}
%  	      	\item Proposed \emph{GOAM} algorithm is to solve research problem
%  	      	\item Utilise pruning strategies to reduce time complexity
%  	      	\end{enumerate}
%%  	      \end{minipage}
%}
%%%%%%%%%% -------------------------------------------------------------------- %%%%%%%%%%


%%%%%%%%%% -------------------------------------------------------------------- %%%%%%%%%%
\DIFdelbegin %DIFDELCMD < \block{Introduction}{
%DIFDELCMD <     Many real world applications call for one important function
%DIFDELCMD <     of identifying the set of features
%DIFDELCMD <     on which the interested object is most distinguished from others.
%DIFDELCMD <     Usually,
%DIFDELCMD <     this object is termed as the query object,
%DIFDELCMD <     and the set of features are referred to as the \emph{subspaces} or \emph{aspects}.
%DIFDELCMD <     Accordingly,
%DIFDELCMD <     this research problem is referred to as
%DIFDELCMD <     \emph{outlying aspects mining},
%DIFDELCMD <     which is different from \emph{outlier detection}.
%DIFDELCMD <   	

%DIFDELCMD <   	\begin{description}
\begin{description}%DIFAUXCMD
%DIFDELCMD <   	\item[Outlying Aspects Mining] aims to identify a subspace
%DIFDELCMD <     which makes the query object most outlying,
%DIFDELCMD <     rather than verifying whether it is an outlier or not.
%DIFDELCMD <     The task of \emph{Outlying Aspects Mining}
%DIFDELCMD <     is to explain which aspects make the query object most different.
%DIFDELCMD <   	

%DIFDELCMD <   	\item[Outlier Detection] aims to identify all possible outliers in the dataset,
%DIFDELCMD <     without explaining why or how they are different.
%DIFDELCMD <     Hence,
%DIFDELCMD <     the outlying aspects mining is also referred to
%DIFDELCMD <     \emph{outlier interpretation}
%DIFDELCMD <     or \emph{object explanation}.

\end{description}%DIFAUXCMD
%DIFDELCMD <   	\end{description}
%DIFDELCMD < 

%DIFDELCMD <   	In this paper,
%DIFDELCMD <     we extend the task of \emph{outlying aspects mining} to the \emph{group} level,
%DIFDELCMD <     formalize the research problem of \emph{group outlying aspects mining},
%DIFDELCMD <     and propose a novel algorithm named GOAM to solve the
%DIFDELCMD <     \emph{group outlying aspects mining} problem.
%DIFDELCMD < }
%DIFDELCMD < %%%
\DIFdelend \DIFaddbegin \block{Introduction}{
  San Francisco was infamous for housing some of the world's most notorious criminals on 
  the inescapable island of Alcatraz. Today, the city is known more for its tech scene than 
  its criminal past. From Sunset to SOMA, and Marina to Excelsior, this project analyzes 12
   years of crime reports from across all of San Francisco's neighborhoods to create a model 
   that predicts the category of crime that occurred, given time and location.

  	\begin{description}
    \item[\DIFadd{The Dataset}]  is in a tabular form and includes chronological, geographical and
     text data and contains incidents derived from the SFPD Crime Incident Reporting system.

    \item[\DIFadd{Data Visualization}] is the main method designed in this project. 
    Through data visualization, the data set we will present will finally be visualized and
     the understanding of the data will be more direct.

  	\end{description}

    The main task of this project is to make a prediction of the type, time and place of crime
     in San Francisco. In this article, we will give an overview of the whole project from data 
     sets, eigenvalues, feature item selection, modeling and conclusion.
}
\DIFaddend %%%%%%%%%% -------------------------------------------------------------------- %%%%%%%%%%


%%%%%%%%%% -------------------------------------------------------------------- %%%%%%%%%%
\DIFdelbegin %DIFDELCMD < \block{Group Outlying Aspects Mining}{
%DIFDELCMD < \begin{itemize}
\begin{itemize}%DIFAUXCMD
%DIFDELCMD <     \item
\item%DIFAUXCMD
%DIFDELCMD <     %\emph{Group Outlying Aspects Mining}
%DIFDELCMD <     It aims to \emph{identify a subset of aspects (or subspace)
%DIFDELCMD <     which makes the query group, rather than the single object,
%DIFDELCMD <     obviously different}.
%DIFDELCMD <     What we are interested in the task of \emph{group outlying aspects mining}
%DIFDELCMD <     is to explain which aspects make the query group distinctive
%DIFDELCMD <     different from the other groups.
%DIFDELCMD < 

%DIFDELCMD <     \item
\item%DIFAUXCMD
%DIFDELCMD <     \emph{Group Outlying Aspects Mining},
%DIFDELCMD <     \emph{Outlying Aspects Mining} and
%DIFDELCMD <     \emph{Outlier Detection} are different with each other.

\end{itemize}%DIFAUXCMD
%DIFDELCMD < \end{itemize}
%DIFDELCMD < 

%DIFDELCMD < \begin{center}
%DIFDELCMD <     \begin{minipage}{0.3\linewidth}
%DIFDELCMD <     \centering
%DIFDELCMD <     \begin{tikzfigure}
%DIFDELCMD <     \missingfigure[figcolor=white]{Testing figcolor}
%DIFDELCMD <     {\small{Group Outlying Aspects Mining}}
%DIFDELCMD <     \end{tikzfigure}%
%DIFDELCMD <     \end{minipage}
%DIFDELCMD <     \hfill
%DIFDELCMD <     \begin{minipage}{0.3\linewidth}
%DIFDELCMD <     \centering
%DIFDELCMD <     \begin{tikzfigure}
%DIFDELCMD <     \missingfigure[figcolor=white]{Testing figcolor}
%DIFDELCMD <     {\small{Outlying Aspects Mining}}
%DIFDELCMD <     \end{tikzfigure}%
%DIFDELCMD <     \end{minipage}
%DIFDELCMD <     \hfill
%DIFDELCMD <     \begin{minipage}{0.3\linewidth}
%DIFDELCMD <     \centering
%DIFDELCMD <     \begin{tikzfigure}
%DIFDELCMD <     \missingfigure[figcolor=white]{Testing figcolor}
%DIFDELCMD <     {\small{Outlier Detection}}
%DIFDELCMD <     \end{tikzfigure}%
%DIFDELCMD <     \end{minipage}
%DIFDELCMD < \end{center}
%DIFDELCMD < }
%DIFDELCMD < %%%
\DIFdelend \DIFaddbegin \block{The Dataset}{
  The data ranges from \emph{1/1/2003 to 5/13/2015} creating a training dataset with nine 
  features and 878,049 samples.
\begin{itemize}
    %\emph{Group Outlying Aspects Mining}

    \item
    \emph{Dates - timestamp of the crime incident}
    \item
    \emph{ Category - category of the crime incident. (This is our target variable.)} and
    \item 
    \emph{  Descript - detailed description of the crime incident}
    \item 
    \emph{  DayOfWeek - the day of the week}
    \item 
    \emph{ PdDistrict - the name of the Police Department District}
    \item 
    \emph{ Resolution - The resolution of the crime incident}
    \item 
    \emph{Address - the approximate street address of the crime incident }
    \item 
    \emph{X - Longitude}
    \item 
    \emph{Y - Latitude}
  \end{itemize}
}
\DIFaddend %%%%%%%%%% -------------------------------------------------------------------- %%%%%%%%%%


%%%%%%%%%% -------------------------------------------------------------------- %%%%%%%%%%

%\note{Note with default behavior}

%\note[targetoffsetx=12cm, targetoffsety=-1cm, angle=20, rotate=25]
%{Note \\ offset and rotated}

 % First column - second block


%%%%%%%%%% -------------------------------------------------------------------- %%%%%%%%%%
\DIFdelbegin %DIFDELCMD < \block{GOAM Algorithm}{
%DIFDELCMD <   	We propose the \emph{GOAM} algorithm to solve the research problem of
%DIFDELCMD <     \emph{Group Outlying Aspects Mining}.
%DIFDELCMD <   	The \emph{GOAM} algorithm includes three major steps.
%DIFDELCMD < %    1) Group Feature Extraction,
%DIFDELCMD < %    2) Outlying Degree Scoring, and
%DIFDELCMD < %    3) Outlying Aspects Identification.
%DIFDELCMD <   	

%DIFDELCMD < \begin{tikzfigure}%[Overall architecture of \emph{GOAM} algorithm]
%DIFDELCMD < %  \includegraphics[width=0.8\linewidth]{figures//framework.pdf}
%DIFDELCMD <     \missingfigure[figcolor=white]{Testing figcolor}
%DIFDELCMD < \end{tikzfigure}
%DIFDELCMD < 		

%DIFDELCMD < \begin{description}
\begin{description}%DIFAUXCMD
%DIFDELCMD <   	\item[Group Feature Extraction]
\item[\DIFdel{Group Feature Extraction}]%DIFAUXCMD
%DIFDELCMD <   	Let $f_1$, $f_2$, $f_3$ represent three features of $G_q$.
%DIFDELCMD <     We count the frequency of each value for one feature.
%DIFDELCMD <     Then use the histogram to represent each feature.
%DIFDELCMD <     Similarly,
%DIFDELCMD <     we can extract other features for each group.
%DIFDELCMD < 

%DIFDELCMD < %    \item
%DIFDELCMD < %    The histogram of $G_q$ on three features are as follows.

\end{description}%DIFAUXCMD
%DIFDELCMD < \end{description}
%DIFDELCMD < 

%DIFDELCMD < \begin{center}
%DIFDELCMD <     \begin{minipage}{0.3\linewidth}
%DIFDELCMD <     \centering
%DIFDELCMD <     \begin{tikzfigure}
%DIFDELCMD <     \missingfigure[figcolor=white]{Testing figcolor}
%DIFDELCMD <     {\small{Histogram of $G_q$ on $f_1$}}
%DIFDELCMD <     \end{tikzfigure}%
%DIFDELCMD <     \end{minipage}
%DIFDELCMD <     \hfill
%DIFDELCMD <     \begin{minipage}{0.3\linewidth}
%DIFDELCMD <     \centering
%DIFDELCMD <     \begin{tikzfigure}
%DIFDELCMD <     \missingfigure[figcolor=white]{Testing figcolor}
%DIFDELCMD <     {\small{Histogram of $G_q$ on $f_2$}}
%DIFDELCMD <     \end{tikzfigure}%
%DIFDELCMD <     \end{minipage}
%DIFDELCMD <     \hfill
%DIFDELCMD <     \begin{minipage}{0.3\linewidth}
%DIFDELCMD <     \centering
%DIFDELCMD <     \begin{tikzfigure}
%DIFDELCMD <     \missingfigure[figcolor=white]{Testing figcolor}
%DIFDELCMD <     {\small{Histogram of $G_q$ on $f_3$}}
%DIFDELCMD <     \end{tikzfigure}%
%DIFDELCMD <     \end{minipage}
%DIFDELCMD < \end{center}
%DIFDELCMD < \begin{description}
\begin{description}%DIFAUXCMD
%DIFDELCMD < \item[Outlying Degree Scoring]
\item[\DIFdel{Outlying Degree Scoring}]%DIFAUXCMD
%DIFDELCMD <     In this step,
%DIFDELCMD <     we first calculate the \emph{earth mover distance} (EMD) of one feature among different groups.
%DIFDELCMD <     The earth mover distance reflects the minimum mean distance
%DIFDELCMD <     between groups on one feature.
%DIFDELCMD <     So,
%DIFDELCMD <     we utilize the EMD to measure the difference between groups of each feature.

\end{description}%DIFAUXCMD
%DIFDELCMD < \end{description}
%DIFDELCMD < }
%DIFDELCMD < %%%
\DIFdelend \DIFaddbegin \block{Feature Item}{
  The dataset contains 2,323 duplicates that are meaningless 
  and should be deleted.We will also use coordinates to calculate the 
  distribution of data points on the map of San Francisco.At the same time, 
  67 incorrect locations were found.
  \\
 \vspace{.5cm}
  \centering
  \begin{tabular}{ c | c  }
    \toprule
   Datas                & datetime64    \\
    \toprule
   Category             & object  \\
   \toprule
   Descript             & object    \\
   \toprule
   DayOfWeek            &object     \\
   \toprule
   PdDistrict           & object \\
   \toprule
   Resolution            &object    \\
   \toprule
   Address              & object    \\
   \toprule
   x          & float64    \\
   \toprule
   Y        & float64       \\
    \bottomrule
\end{tabular}
\vspace{.2cm}
  \begin{description}

    \item[\DIFadd{Dates \& Day of the week}]
    These variables are distributed uniformly between 1/1/2003 
    to 5/13/2015 (and Monday to Sunday) and split between the training 
    and the testing dataset as mentioned before. We did not notice any 
    anomalies on these variables.
    The median frequency of incidents is 389 per day with a standard deviation
     of 48.51.
     \\
    \item [Per Week]
    Also, there is no significant deviation of incidents frequency throughout 
    the week. Thus we do not expect this variable to play a significant role 
    in the prediction.
    \\

   
  \end{description}

  	
%    1) Group Feature Extraction,
%    2) Outlying Degree Scoring, and
%    3) Outlying Aspects Identification.

}
\DIFaddend %%%%%%%%%% -------------------------------------------------------------------- %%%%%%%%%%


% SECOND column
\column{0.5}
 %Second column with first block's top edge aligned with with previous column's top.

%%%%%%%%%% -------------------------------------------------------------------- %%%%%%%%%%
\DIFdelbegin %DIFDELCMD < \block{GOAM Algorithm}{
%DIFDELCMD < \begin{description}
\begin{description}%DIFAUXCMD
%DIFDELCMD <     \item
\item%DIFAUXCMD
%DIFDELCMD <     Second,
%DIFDELCMD <     based on the \emph{earth move distance},
%DIFDELCMD <     we calculate the outlying degree.

\end{description}%DIFAUXCMD
%DIFDELCMD < \end{description}
%DIFDELCMD < 

%DIFDELCMD < \begin{tikzfigure}%[Overall architecture of \emph{GOAM} algorithm]
%DIFDELCMD <     \missingfigure[figcolor=white]{Testing figcolor}
%DIFDELCMD < \end{tikzfigure}
%DIFDELCMD <   where $G_q$ is the query group,
%DIFDELCMD <   $n$ is the number of compare groups,
%DIFDELCMD <   and $h_{k_s}$ is the histogram representation of $G_k$ in the subspace $s$.
%DIFDELCMD < 

%DIFDELCMD < \begin{description}
\begin{description}%DIFAUXCMD
%DIFDELCMD <   	\item[Outlying Aspects Identification]
\item[\DIFdel{Outlying Aspects Identification}]%DIFAUXCMD
%DIFDELCMD <     In this step,
%DIFDELCMD <     based on the value of outlying degree
%DIFDELCMD <     we will identify the group outlying aspects.
%DIFDELCMD <     If a feature's outlying degree is greater than a threshold,
%DIFDELCMD <     the more likely the feature is group outlying aspect.
%DIFDELCMD <     When the dimensionality of features is high,
%DIFDELCMD <     we adopt a stage-wise candidate subspace construction strategy to
%DIFDELCMD <     alleviate the exponential explosion.

\end{description}%DIFAUXCMD
%DIFDELCMD < \end{description}
%DIFDELCMD < }
%DIFDELCMD < %%%
\DIFdelend \DIFaddbegin \block{Feature Item}{
\begin{description}
  \item [Category]
    There are 39 discrete categories that the police department file the 
    incidents with the most common being Larceny/Theft (19.91\%), Non/Criminal
     (10.50\%), and Assault(8.77\%).
     \\
  \item [Police District]
  There are significant differences between the different districts of the 
  City with the Southern district having the most incidents (17.87\%) 
  followed by Mission (13.67\%) and Northern (12.00\%).
  \\
  \item [X - Longitude Y - Latitude]We have tested that the coordinates 
    belong inside the boundaries of the city. Although longitude does not 
    contain any outliers, latitude includes some 90o values which correspond 
    to the North Pole.
    \\
    \item [Address]as a text field, requires advanced techniques to use it for 
    the prediction. Instead in this project, we will use it to extract if the 
    incident has happened on the road or in a building block.
\end{description}

\begin{tikzfigure}%[Overall architecture of \emph{GOAM} algorithm]
    \missingfigure[figcolor=white]{Testing figcolor}
\end{tikzfigure}
}
\DIFaddend %%%%%%%%%% -------------------------------------------------------------------- %%%%%%%%%%
% Second column - first block


%%%%%%%%%% -------------------------------------------------------------------- %%%%%%%%%%
\DIFdelbegin %DIFDELCMD < \block[titleleft]{Experiment}
%DIFDELCMD < %%%
\DIFdelend \DIFaddbegin \block[titleleft]{Data Visualization}
\DIFaddend {
  \DIFdelbegin %DIFDELCMD < \begin{description}
\begin{description}%DIFAUXCMD
%DIFDELCMD <   	\item[Synthetic Dataset] %%%
\item[\DIFdel{Synthetic Dataset}]%DIFAUXCMD
\DIFdel{contains $10$ groups and $8$ features.
    Each group consists of $10$ members, and each member has $8$ features. }
\end{description}%DIFAUXCMD
%DIFDELCMD < \end{description}
%DIFDELCMD < %%%
\DIFdelend \DIFaddbegin \DIFadd{Based on the Project’s statement, we need to predict the probability
   of each type of crime based on time and location. That being said, 
   we present two diagrams to visualize the importance of these variables. 
   The first one presents the geographic density of 9 random crime
    categories. We can see that although the epicenter of most of the 
    crimes resides on the northeast of the city, each crime has a different 
    density on the rest of the city. This fact is a reliable indication
     that the location ( coordinates / Police District) will be a significant
      factor for the analysis and the forecasting.
}\DIFaddend \vspace{.5cm}
\DIFdelbegin %DIFDELCMD < \begin{tabular}{ c | c | c | c }
%DIFDELCMD <     \toprule
%DIFDELCMD <     %%%
\DIFdel{Method     }%DIFDELCMD < &  %%%
\DIFdel{Truth Outlying Aspects    }%DIFDELCMD < & %%%
\DIFdel{Identified Aspects }%DIFDELCMD < & %%%
\DIFdel{Accuracy      }%DIFDELCMD < \\
%DIFDELCMD <     \midrule
%DIFDELCMD <     %%%
\DIFdel{GOAM       }%DIFDELCMD < &  %%%
\DIFdel{$\{F_1\}$, 
   $\{F_2F_4\}$   }%DIFDELCMD < &  %%%
\DIFdel{$\{F_1\}$, $\{F_2F_4\}$    }%DIFDELCMD < & %%%
\DIFdel{100\%    }%DIFDELCMD < \\
%DIFDELCMD < %%%
\DIFdelend 

\DIFdelbegin \DIFdel{Arithmetic Mean based OAM }%DIFDELCMD < &  %%%
\DIFdel{$\{F_1\}$, $\{F_2F_4\}$   }%DIFDELCMD < &  %%%
\DIFdel{$\{F_4\}$, $\{F_2\}$    }%DIFDELCMD < &  %%%
\DIFdel{0\% }%DIFDELCMD < \\
%DIFDELCMD < 

%DIFDELCMD <      %%%
\DIFdel{Median based OAM }%DIFDELCMD < &  %%%
\DIFdel{$\{F_1\}$, $\{F_2F_4\}$   }%DIFDELCMD < &  %%%
\DIFdel{$\{F_2\}$, $\{F_4\}$    }%DIFDELCMD < &           %%%
\DIFdel{0\% }%DIFDELCMD < \\
%DIFDELCMD <      \bottomrule
%DIFDELCMD < \end{tabular}
%DIFDELCMD < %%%
\DIFdelend \vspace{.2cm}
\begin{description}
    \DIFdelbegin %DIFDELCMD < \item
\item%DIFAUXCMD
%DIFDELCMD <     %%%
\DIFdel{It can be observed that the GOAM method can identify the 
    trivial outlying features
    and non-trivial outlying subspaces correctly and is obvious from the table
    that the accuracy of GOAM is the best, which is ($100\%$) .
}\DIFdelend \DIFaddbegin \item
    \DIFadd{The diagram presents the average number of incidents per
     hour for five of the crimes' categories. 
    It is evident that different crimes have different frequency 
    during different times of the day. Some examples are that 
    prostitution picks during the evening and all through the night, Gambling incidents start late at night until the morning and Burglary picks early in the morning until the afternoon. As before these are sharp pieces of evidence that the time parameters will have a significant role also.
}\DIFaddend \end{description}


\DIFdelbegin %DIFDELCMD < \begin{description}
\begin{description}%DIFAUXCMD
%DIFDELCMD < \item[NBA Dataset] %%%
\item[\DIFdel{NBA Dataset}]%DIFAUXCMD
\DIFdel{was collected from Yahoo Sports
website (}%DIFDELCMD < \url{http://sports.yahoo.com.cn/nba}%%%
\DIFdel{).
The data include all teams from the six divisions, and each player in the team has $12$ features. }
\end{description}%DIFAUXCMD
%DIFDELCMD < \end{description}
%DIFDELCMD < %%%
\DIFdelend \vspace{.5cm}     
\DIFdelbegin %DIFDELCMD < \begin{tabular}{ c | c | c }
%DIFDELCMD <     \toprule
%DIFDELCMD <     %%%
\DIFdel{Teams                   }%DIFDELCMD < & %%%
\DIFdel{Trivial Outlying Aspects  }%DIFDELCMD < & %%%
\DIFdel{NonTrivial Outlying Aspects    }%DIFDELCMD < \\
%DIFDELCMD <     \toprule
%DIFDELCMD <     %%%
\DIFdel{Cleveland Cavaliers     }%DIFDELCMD < & %%%
\DIFdel{\{3FA\}                   }%DIFDELCMD < & %%%
\DIFdel{\{FGA, FT\%\}, \{FGA, FG\%\} }%DIFDELCMD < \\
%DIFDELCMD <     %%%
\DIFdel{Orlando Magic           }%DIFDELCMD < & %%%
\DIFdel{\{Stl\}                   }%DIFDELCMD < & %%%
\DIFdel{None                         }%DIFDELCMD < \\
%DIFDELCMD <     %%%
\DIFdel{Milwaukee Bucks         }%DIFDELCMD < & %%%
\DIFdel{\{To\}, \{FTA\}           }%DIFDELCMD < & %%%
\DIFdel{\{FGA, FTA\}, \{3FA, FTA\}     }%DIFDELCMD < \\
%DIFDELCMD < %%%
%DIF <     Golden State Warriors   & \{FG\%\}                  & \{FT\%, Blk\}, \{FGA, 3PT\%, FTA\}\\
%DIF <     Utah Jazz               & \{Blk\}                   & \{3FA, 3PT\%\}                    \\
    \DIFdel{New Orleans Pelicans    }%DIFDELCMD < & %%%
\DIFdel{\{FT\%\}, \{FTA\}         }%DIFDELCMD < & %%%
\DIFdel{\{FTA, Stl\}, \{FTA, To\}          }%DIFDELCMD < \\
%DIFDELCMD <     \bottomrule
%DIFDELCMD < \end{tabular}
%DIFDELCMD <            

%DIFDELCMD < %%%
\DIFdelend \begin{minipage}{0.5\linewidth}
    \centering
    \begin{tikzfigure}
    \missingfigure[figcolor=white]{Testing figcolor}

    {\small{New Orleans Pelicans on FT\%}}
    \end{tikzfigure}%
\end{minipage}
\hfill
\begin{minipage}{0.5\linewidth}
    \centering
    \begin{tikzfigure}
    \missingfigure[figcolor=white]{Testing figcolor}

    {\small{New Orleans Pelicans on FTA}}
    \end{tikzfigure}%
\end{minipage}
\vspace{.2cm}
\DIFdelbegin %DIFDELCMD < \begin{description}
\begin{description}%DIFAUXCMD
%DIFDELCMD < \item
\item%DIFAUXCMD
%DIFDELCMD < %%%
\texttt{\DIFdel{New Orleans Pelicans}} %DIFAUXCMD
\DIFdel{has more players with
lower \{free throw percentage\}, \{free throws attempted\}.
}
\end{description}%DIFAUXCMD
%DIFDELCMD < \end{description}
%DIFDELCMD < %%%
\DIFdelend }
%%%%%%%%%% -------------------------------------------------------------------- %%%%%%%%%%


% Second column - second block
%%%%%%%%%% -------------------------------------------------------------------- %%%%%%%%%%
\block[titlewidthscale=1, bodywidthscale=1]
{Conclusion}
{
\begin{description}
  \item[Problem Definition]
  \DIFdelbegin \DIFdel{Formalize the problem of Group Outlying Aspects Mining by extending outlying aspects mining.
}%DIFDELCMD < 

%DIFDELCMD <   \item[GOAM algorithm]
\item[\DIFdel{GOAM algorithm}]%DIFAUXCMD
%DIFDELCMD <   %%%
\DIFdel{Propose GOAM algorithm to solve the }\emph{\DIFdel{Group}}%DIFAUXCMD
\DIFdelend \DIFaddbegin \DIFadd{To examine the specific problem, we will apply a full Data Science life cycle composed of the six steps.
  }\DIFaddend \\
  \DIFdelbegin \emph{\DIFdel{Outlying Aspects Mining}} %DIFAUXCMD
\DIFdel{problem.
  }%DIFDELCMD < 

%DIFDELCMD <   \item[Strategies]
\item[\DIFdel{Strategies}]%DIFAUXCMD
%DIFDELCMD <   %%%
\DIFdel{Utilize the pruning strategies to }%DIFDELCMD < \\ %%%
\DIFdel{reduce time complexity}\DIFdelend \DIFaddbegin \DIFadd{In this paper, we did not complete the design of this project in 
  strict accordance with the whole journey, but through the 
  main data visualization, we understand and analyze the data set provided visually, which is a great progress}\DIFaddend .
\end{description}
}
%%%%%%%%%% -------------------------------------------------------------------- %%%%%%%%%%


% Bottomblock
%%%%%%%%%% -------------------------------------------------------------------- %%%%%%%%%%
\colorlet{notebgcolor}{blue!20}
\colorlet{notefrcolor}{blue!20}
\note[targetoffsetx=8cm, targetoffsety=-4cm, angle=30, rotate=15,
radius=2cm, width=.26\textwidth]{
Acknowledgement
\begin{itemize}
    \item
    International Cooperation Project (Y7Z0511101)
    of IIE,
    Chinese Academy of Sciences
 \end{itemize}
}

%\note[targetoffsetx=8cm, targetoffsety=-10cm,rotate=0,angle=180,radius=8cm,width=.46\textwidth,innersep=.1cm]{
%Acknowledgement
%}

%\block[titlewidthscale=0.9, bodywidthscale=0.9]
%{Acknowledgement}{
%}
%%%%%%%%%% -------------------------------------------------------------------- %%%%%%%%%%

\end{columns}


%%%%%%%%%% -------------------------------------------------------------------- %%%%%%%%%%
%[titleleft, titleoffsetx=2em, titleoffsety=1em, bodyoffsetx=2em,%
%roundedcorners=10, linewidth=0mm, titlewidthscale=0.7,%
%bodywidthscale=0.9, titlecenter]

%\colorlet{noteframecolor}{blue!20}
\colorlet{notebgcolor}{blue!20}
\colorlet{notefrcolor}{blue!20}
\note[targetoffsetx=-13cm, targetoffsety=-12cm,rotate=0,angle=180,radius=8cm,width=.96\textwidth,innersep=.4cm]
{
\begin{minipage}{0.3\linewidth}
\centering
\includegraphics[width=24cm]{logos/tulip-wordmark.eps}
\end{minipage}
\begin{minipage}{0.7\linewidth}
{ \centering
 The $11^{th}$ International Conference on Knowledge Science,
  Engineering and Management (KSEM 2018),
  17-19/08/2018, Changchun, China
}
\end{minipage}
}
%%%%%%%%%% -------------------------------------------------------------------- %%%%%%%%%%


\end{document}

%\endinput
%%
%% End of file `tikzposter-template.tex'.
